\documentclass{article}
\usepackage[margin=0.5in]{geometry}
 \usepackage[table,xcdraw]{xcolor}
\begin{document}

	\title{THE REPORT ABOUT COMPUTER SCIENCE STUDENT’S ATTITUDE TO THE USAGE OF PERSONAL MOBILE PHONE IN THE LECTURE}
	\author{Author:OBBO PETER}
	\date{Reg no:15/U/1053 \\  Students no:21500965}
	
	
\maketitle

\tableofcontents

\section{ABSTRACT}\label{sec:into}

The aim of this report was to investigate computer science students’ attitudes to personal mobile phone use in the lecture. The survey on the attitudes towards the use of the mobile phone during the lecture was conducted. The results indicate that the majority of students use these phones. The report concludes that personal mobile phones are disruptive in the lecture and they should be turned off in the lecture. It also recommends that University policy of banning the use of mobile phones except in the exceptional circumstances. 

\section{INTRODUCTION}\label{sec:into}
There has been a massive increase in the use of personal mobile phone over the past eight years and there is every indication that this will continue. According to BLESSEAD (2017) by 2020 almost all computer science students in the university will carry personal mobile phones. BLESSEAD describes this phenomenon as serious in the extreme, potentially undermining the foundation of communication in our society. Currently at CIT 85\% of computer science students have phones.


Recently a number of lecturers complained about the use of the personal mobile phones during the lecture and asked what the official policy is. At present the is no official university policy regarding phone usage in the lecture. This report examines the issue of mobile phone usage in lecture not anywhere else at campus at any time.


For the purpose of this report a personal mobile phone is a personally funded phone for private calls as opposed by any student.


\section{METHODS}\label{sec:into}
This research was conducted by questionnaire and investigated computer science students’ attitudes to the use of mobile phone in lecture. A total of 200 questionnaires were distributed and they used Lekert scales to assess social attitudes to mobile phone usage and provided open ended responses for additional comments.

\section{RESULTS}\label{sec:into}
There was an 85\% response rate to the questionnaire. A breakdown of the responses is listed below in the Table 1. It can be clearly seen from the result that mobile phones are considered to be disruptive and should be turned off during  lectures.

This survey also allowed students to identify weather mobile phones should be allowed in lectures and also assessed them towards receiving personal phone calls in the lecture in open ended question. These results showed that students thought that in some circumstances like medical or emergencies, receiving personal phone calls was acceptable, but generally receiving personal phone calls was not necessary.


\begin{table}[]
\centering
\caption{breakdown of the responses }
\label{my-label}
\begin{tabular}{|l|l|l|l|l|}
\hline
Personal mobile phone usage during lectures & Strongly agree    \% & Agree \% & Disagree \% & Strongly disagree \%     \\ \hline
Not a problem                               & 5                    & 7        & 65          & 23                       \\ \hline
An issue                                    & 40                   & 45       & 10          & 5                        \\ \hline
Disruptive                                  & 80                   & 10       & 7           & {\color[HTML]{FFFFFF} 3} \\ \hline
Phones should be permissible                & 6                    & 16       & 56          & 22                       \\ \hline
Phones should be turned off                 & 85                   & 10       & 3           & 2                        \\ \hline
Allowed in some circumstances               & 10                   & 52       & 24          & \textbf{14}              \\ \hline
\end{tabular}
\end{table}


\section{INTERPRETATION OF THE RESULTS}\label{sec:into}
It can be seen from the results in Table 1 that personal mobile phone use is considered
to a problem; however, it was acknowledged that in some situations it should be
permissible. 80\% of recipients considered mobile phones to be highly disruptive and
there was strong support for phones being turned off during the lecture (85\%). Only 12\%
thought that mobile phone usage in lectures was not a problem,
whereas 85\% felt it was an issue. The results are consistent throughout the survey.
Many of the respondents ,62\%  felt that in exceptional circumstances mobile phones
should be allowed, like medical, but there should be protocols regarding this.



\section{CONCLUSION}\label{sec:into}
The use of mobile phones in lectures is clearly disruptive and they should be
switched off. Most students felt it is not necessary to receive personal phone calls in during lectures except under certain circumstances, but permission should first be sought
from the lecturer or any instructor.



\section{RECOMMENDATION}\label{sec:into}
It is recommended that University develops an official policy regarding the use of mobile
phones during lectures. The policy should recommend:

	\begin{itemize}
   	  \item Mobile phones are banned in lectures.
	  \item Mobiles phone may be used in exceptional circumstances but only with the
permission of the appropriate lecturer or any instructor.
	\end{itemize}


Finally, the policy needs to apply to all computer science students in the university.


\end{document}